The conclusion may contain facts discovered throught the comparision of solving times. Thoughs about intensification and diversification of the algorithms. Parametrizations that work and why the do, etc..

In this project has been developed an ILP model and two meta-heuriscs in order to find out the optimal (for ILP) and good (for meta-heuristics) nurse schedules in a hospital taking into account several constraints.

On the first hand, the Integer Linear Programming part has been really challenging to end up with a really simplified model that takes into account every constraint. [More here!]

On the second hand, the meta-heuristics has been also a challenging to end up with a reasonable greedy function, that is in fact the core of all applied meta-heuristics here. But the more challenging part has been to get a good inputs for the experiments, specially the largest of them. It has been so difficult since the complexity is not only a matter of number of nurses but also another factors that determine the problem context, but finally we end up with good inputs by tunning the instance generator.

Finally, the results has been satisfactory since they are approximately what we can expect from the applied techniques. [More here!]

\pagebreak